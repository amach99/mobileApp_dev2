\documentclass{article}
\usepackage[utf8]{inputenc}

\title{Week 3 Bibliography Report}
\author{Alok Macharla }
\date{February 22, 2021}

\begin{document}

\maketitle

\section{Source 1}
This source is about a company called Mojo Vision which is working on a contact lens prototype that would contain an image sensor, display, motion sensors, and wireless radios all contained within the lens, to enable an immersive virtual reality experience. 
Virtual reality is the next big frontier which can  combine the real world we live in with the creative personalized online world we have forged on our laptops or gaming systems. Virtual Reality gives us the opportunity to get the best of both worlds. With so much research and development being done in this field, and increasing amount of innovative products like Mojo Vision's VR contact lens' will be developed to help us interact with the virtual world. Furthermore thee advancements can help in other non tech industries like medicine, where the development of these contact lens could help assist those with vision impairments. 
Article: \cite{perry_2021}

\section{Source 2}
This source discusses what could be the next big innovation after smartphones. When smartphones originally exploded in the consumer market in the last decade, they started a whole new ecosystem of technology that was geared around the smartphone users. Since then there has been many improvements to the handheld device and now it is an essential part of everyday life. However smartphone sales have been dropping over the last couple years, and as a result the search for the next big consumer product has begun. The development of Augmented Reality or AR technology has opened the door to a new type of device that can enable the user to interact with virtual reality. The rise of wearable tech like smart watches and smart glasses can be taken advantage of to create an even more immersive experience that will change the way we interact with our surroundings.
Article: \cite{leswing_2021}

\section{Source 3}
This source is about a breakthrough discovery in Photonic research by the North Carolina State University. Engineering researchers at the University have developed a new design for photonic devices that allows them to control the direction and polarization of light using thin-film LEDs. The development of hardware is just as crucial as the development of software, and these researchers claim that their discovery makes smaller, more efficient virtual reality technology a possibility. Their new design for photonic devices can help create lighter and less bulky AR headsets, as well as increase the amount of light which enters to help provide a clearer view of the real world while integrating AR software into the user's field of view.
Article: \cite{edsgcl.65176928220210221}

\bibliographystyle{plain}
\bibliography{sources}
\end{document}
