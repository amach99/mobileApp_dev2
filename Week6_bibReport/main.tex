\documentclass{article}
\usepackage[utf8]{inputenc}

\title{Week 6 Bibliography Report}
\author{Alok Macharla}
\date{March 2021}

\begin{document}

\maketitle

\section{Automation's impact on Mental Health}
This source is about the impact of automation on mental health. Due to many people losing their jobs to automation, this article predicts that the decrease of job security will have a negative consequences on mental health. \cite{PATEL201854}

\section{Automation's impact on Job Satisfaction}
This article discusses how the rise of automation will result in a decrease in job satisfaction among low and high skilled workers. This article also predicts that lower skilled workers are more likely to experience a decrease in job satisfaction because they are more exposed to automation and at risk of losing their jobs. However even high skilled workers have concerns with job security and the rise of automation which impacts their performance negatively as well. \cite{3325322720201130}

\section{The Rise of Autonomous Vehicles}
This source discusses the implications of the use of autonomous vehicles in the workplace.
The authors identify 3 main points that drive the increase in autonomous vehicles:
(1) the inevitability of the universal use of AVs and hence the immediate need for labour market planning, (2) associated potential effects on occupations that are not primarily structured around driving, and (3) the possibility of increased worker safety and enhanced commuting opportunities.\cite{3020020620180830}

\section{Government's response to Automation}
This source talks about the efforts of a NY senator who is pushing for a government program which helps workers who have lost/ will lose their jobs due to the rise of automation. This is one example of how government must reach in order to stop mass unemployment due to automation. The program will also help displaced workers receive retraining to help them Learn skills required for the high tech positions of the future. \cite{edsgit.A54253820120180613}

\section{Counter Argument to Automation will result in Job Loss}
This source argues against the 'misconception" that the rise of automation will result in catastrophic job loss. The author believes that as long as our society is made up of social beings, there will always be an emphasis on their well being over a complete blind rush to an autonomous market. \cite{2832185620170301}

\section{Automation's impact on Job Growth Trends}
This source discusses a trend in job growth/loss sparked by automation. Automation initially spurs job growth because demand is high, however as demand decreases so does the number of jobs. The authors theorize that while automation may not cause mass unemployment, it will definitely cause workers to make serious changes such as transitioning to new occupations. \cite{14450544920191001}

\bibliographystyle{plain}
\bibliography{sources}
\end{document}
