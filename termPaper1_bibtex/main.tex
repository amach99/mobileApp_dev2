\documentclass{article}
\usepackage[utf8]{inputenc}

\title{The Impact of Mobile Application Development on Commerce}
\author{Alok Macharla }
\date{March 2, 2021}

\begin{document}
\maketitle

\section{Abstract}
Before the development of mobile applications that helped promote commerce over the past decade, businesses struggled to grow and overcome hurdles such as the high cost of marketing, a technology barrier that prevented vendors from serving customers, and an inefficient method of currency exchange.

\section{Introduction}
Although we still only have 24 hours in a day, the world is moving at a faster pace than ever before. Largely due to the rapid development of technology which has had a significant impact on every major industry around the world. The focus of this paper is specifically the impact of mobile application development on commerce. Over the past decade the development of mobile applications has significantly impacted commerce by revolutionizing the way currency can be exchanged and enabling businesses to reach a wider audience. Mobile applications like Venmo and Cash App provided us with a new way to exchange currency in return for goods or services that was fast and convenient. Companies like Square inc. and Apple reinvented traditional hardware used in a day-to-day commerce which enabled previously cash-only businesses to grow and develop an online retail presence. While social media websites like Facebook and Twitter released mobile applications that transformed them into 24/7 personalized advertisement boards. This provided businesses with a platform to connect with a larger online audience. After a more in-depth analysis of how mobile applications development impacted commerce, this paper concludes with a brief discussion on any negative implications of the rise of mobile applications, followed by a closing statement summarizing the main points of the paper.

\section{The Revolution of Currency Exchange}
Venmo and Cash App are two of the most popular apps downloaded from Apple's app store and Google's play store. They enable millions of people to send money to each other and have become essential services as people are less inclined to write a cheque or visit an ATM in today's fast paced world. Venmo was founded by Iqram Magdon-Ismail and Andrew Kortina who were college roommates at the University of Pennsylvania. Their company was eventually acquired by PayPal in 2013 and has grown immensely in popularity becoming a commonly used verb synonymous with pay, or the act of payment. For example, “Can you venmo me?”.  Apps like Venmo are a growing trend called peer to peer payment services that allow users to use their smartphone to pay for goods or services \cite{bloomenthal_2020}. Cash App is one of the more creative peer to peer payment services on the market and is owned by Square inc. It has grown to offer a plethora of services including commission free investing in the stock market and crypto currency. Also, Cash App was one of the first peer to peer payment services to provide users with a physical debit card that allowed them to access funds in their Cash App account to pay for goods or services \cite{eckstein_2020}. A feature that competitors like Venmo, Paypal, and Apple Cash have all adopted. Venmo and Cash App are just a few examples of how mobile applications have redefined how and where commerce can be done by empowering people with a quick and reliable way to exchange currency with each other.

\section{The Reinvention of Traditional Hardware}
Another way the development of mobile applications changed commerce was the reinvention of traditional hardware used to execute daily transactions. The cash register was first invented in 1879 by James Ritty and with the help of his brother John Ritty, James patented the design in 1883. There have been some notable changes since like the addition of a roll of paper for record keeping purposes in 1984, and in more recent year’s minor technological additions such as credit card scanners and barcode scanners. But none as impactful as Square inc.’s complete reinvention of the cash register’s traditional design \cite{sorensen_2019}.
Founded in 2009 by Jack Dorsey and Jim McKelvey, Square’s first product was called the Square Reader. A small card reader that attaches to a phone via a 3.5 mm headphone jack. With a pretty unremarkable design, just a small black square, this product completely changed how cash-only businesses like farmers markets and contractors were able to run their operations. Vendors who previously struggled with point-of-sale services such as calculating change or keeping a record of every transaction for internal/external bookkeeping purposes were now able to use Square’s products and mobile application to automate these services. Square mobile card reader attachment helped traditional farmers markets evolve by enabling them to accept customers with different forms of payment \cite{cussen_2020}. Later in Square's growth they partnered with Apple to release the Square Stand in June 2013. The stand was compatible with Apple's iPad products and turned the tablet into a more complete point of sale system. This was the first product in a line of stand-alone point of sale systems which were intended to completely replace the traditional cash register. In October 2018 Square released a fully functional cash register that featured a touch screen display, was able to print receipts, and had a built in NFC reader that enabled it to accept all forms of payment including swipes, chips, and most importantly contactless payments.
Apple is another company which played a significant role in reinventing traditional hardware commonly used in commerce. In 2014, Apple released Apple Pay, a mobile contactless payment system and digital wallet service. It essentially digitized bank cards that were traditionally used to withdraw cash at an ATM or pay for goods at a store. This was a crucial innovation for the evolution of peer-to-peer payment services as Apple Pay allowed people to quickly access their funds even if they potentially forgot their wallet. Furthermore, Apple Pay utilizes Near Field Communication (NFC) technology to provide an added layer of security from identify theft in high traffic areas where credit card fraud was a risk \cite{larkin_2020}. Companies like Square and Apple have changed our expectations on how commerce can be conducted by reinventing hardware like the cash register and bank cards that were previously used to process transactions.

\section{The Evolution of Social Media}
Finally, the cost of marketing was a big hurdle that the development of mobile applications inadvertently helped businesses overcome. Before the rise of smartphones and the evolution of social media websites into 24/7 personalized advertisement boards, running a marketing campaign was a very expensive undertaking. Businesses had to first design an ad campaign then spend anywhere from 650 dollars per month to 2.4 million dollars per month (based on size and location) on a billboard hoping that local foot traffic saw their ad. A time-consuming process that did not yield guaranteed results \cite{siemasko_2013}. The other popular medium that was used to deliver advertisements was television. While TV ads did guarantee a wider audience would view your commercial, the exuberant costs of reserving even 30 seconds of airtime was a large deterrent for smaller businesses. Larger corporations could afford to spend an average of 115,000 dollars for half a minute of airtime, but it was impossible for a local restaurant or vendor to raise that kind of capital just for marketing. Fortunately, the evolution of social media websites like Facebook and Twitter was the great equalizer, as technology has often been described. Facebook and Twitter helped business increase brand awareness without spending thousands of dollars on expensive marketing campaigns. Facebook was founded in February 2004 by Mark Zuckerberg and has since grown to become the most downloaded mobile application in the world from 2010-2020 \cite{miller_2019}. What originally started as an online platform for friends to interact and share creative self-expressions, slowly evolved into a free marketing platform. Facebook tracks the likes, follows, and friends of each user to offer a personalized news feed catered to their interests. This enabled businesses to connect directly with their customers and increase the number of new customers who found them. Users could set up a public page for their business to provide customers with a source of information about their products, their store location, and other important details to increase brand awareness \cite{johnston_2021}. Another popular social media service that helped businesses grow is Twitter. An online micro-blogging and social networking service founded by Jack Dorsey, Noah Glass, Biz Stone, and Evan Williams. Its users can post and interact with each other with messages called tweets that have a 280 Character limit \cite{reiff_2021}.
Similar to Facebook, Twitter also tracks user data to provide a personalized news feed. But Twitter’s informal conversational user interface slightly changed its purpose as businesses began using it to release updates for their consumers and were able to receive invaluable live feedback in return. A process that would have been expensive and time consuming to complete using traditional market research methods such as surveys or focus groups. The transformation of social media websites like Facebook and Twitter into a free marketing platform provided businesses with a tool that had a far greater reach than any billboard or television station and was much cheaper compared to the cost of using traditional marketing mediums. 
The reinvention of the cash register, the digitization of bank cards, and the unprecedented reach of social media helped businesses overcome obstacles such as the high cost of marketing and a technology barrier which prevented merchants from processing digital transactions. These developments promoted commerce on a global scale, which in turn increased competition and innovation.

\section{Consequences of Mobile App Development}
Unfortunately, not all business benefited from the development of mobile applications. Traditional department stores like J.C. penny and Toys'R'us have struggled in recent years due to the increase in competition. With the latter, one of the largest toy companies in American History going out of business in September 2017. Toys'R'us was established in 1957 by George Lazarus. Mr. Lazarus was great at marketing and took advantage of the end of World War II. He wanted to show people a contrast to the years of economic depression that the country had undergone during the war.  And that idea became the company’s claim to fame. Toys'R’us was a one of a kind "supermarket for toys”. The stores provided customers in the 1960’s with a sense of awe, as they overwhelmed by the size of the stores and the displays of shelves filled with toys. Mr. Lazarus used jingles and ads to grow brand awareness, but most importantly to influence customers to visit the stores. In the end the cornerstone of the Toys'R'us brand, their massive stores, were the downfall of the company. Toys'R'us is the perfect example of a business that did not adapt fast enough to keep up with the increase in competition and innovation brought by e-commerce at the start of the century. When customers began shifting to online retail in favor of cheaper prices and a more convenient shopping experience, Toys’R’us started to accumulate massive amounts of debt to keep the business running. The main downfall of Toys’R’us was caused by its lack of online retail presence. Since the entire business model was centered around getting customers to visit their stores, the enormous amounts of money they had to spend on marketing campaigns and the overhead costs of keeping stores open became a drain on their resources. Towards the end most of the company’s income went directly towards paying off their debts, which was an estimated 5 billion dollars. So, while Toys’R’us was technically still making money they had to pay roughly 400 million dollars a month in interest fees. This downward decline started in the early 2000’s when their business deal with Amazon fell apart. In 1999, Toys’R’us had set up a 10 year distribution deal with Amazon for them to exclusively sell their toys on the online e-commerce platform. However, after only a couple of years into the deal, Amazon grew dissatisfied with the quality and supply of toys they were receiving from the toy giant. Amazon then went back on the deal and begun selling toys from other companies on its website as well. Toys’R’us responded by filing a lawsuit against Amazon and won 56 million dollars, but consequently ended the partnership. This was a huge blow to the toy company because instead of working on their own digital platform, Toys'R'us relied heavily on Amazon to distribute their toys. So, when the partnership ended Toys’R’us was left without any digital presence or platform to connect with their customers. In the end, the enormous costs of their interest payments and the increase in competition due to the rise of e-commerce proved too much for Toys’R’us to recover from. Unable to retain market share as toy sales continued to decline, Toys’R’us declared bankruptcy in 2017 \cite{deoca_2020}. 

\section{Conclusion}
Commerce is defined as the exchange of goods and services. And over the past decade the development of mobile applications has redefining our expectations of where and how commerce can be conducted. The rise of peer-to-peer payment systems like Venmo and Cash App allowed for a fast and convenient method to exchange money. The complete reinvention of cash registers and bank cards enabled businesses to accept all forms of payment, helping commerce thrive. And the evolution of social media into a powerful marketing platform helped businesses overcome the high costs of traditional marketing mediums by providing them a way to connect with customers and grow their brand through social media posts and promotions.
As commerce continues to evolve to adapt to the times, like with contactless pick-up procedures for take-out in response to the Covid-19 pandemic, mobile applications will continue to play a significant role in our economy.

\bibliographystyle{plain}
\bibliography{sources}
\end{document}

