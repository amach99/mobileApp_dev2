\documentclass{article}
\usepackage[utf8]{inputenc}

\title{The Impact of Automation on the American Workforce}
\author{Alok Macharla }
\date{March 13, 2021}

\begin{document}
\maketitle

\section{Introduction}
Over the history of our society, we have undergone periods of rapid development known as Industrial Revolutions. These revolutions had a significant impact on our economy as new technologies improved traditional business operations to drive innovation and increase prosperity. However, these periods of economic prosperity have historically had some negative consequences as well. As with all innovation, new technology makes previous methods obsolete, and every revolution has created many new jobs to replace the old ones. Currently we are undergoing the fifth industrial revolution in history and its consequences on our job market could be devastating \cite{Popkova2019}. Driven by the rise of industrial automation a large percentage of the population are projected to lose their jobs, resulting in widespread unemployment and poverty. Automation poses a great threat to our society because it impacts the most common jobs in our country and displaces a population that does not have access to the resources needed to learn new skills which are required to transition into the technical occupations of the future.

\section{The Widespread Impact of Automation}
Roughly half of the United States’ current workforce is employed in one of the following five industries: Administrative/Clerical, Retail/Sales, Food prep/Food service, Transportation, and Manufacturing. Automation has already started to have an adverse impact on each of these industries. Stores and Restaurants have started using self-checkout and ordering kiosks that eliminate the need for human cashiers. Call center work has been slowly transitioning towards fully automated natural language processing software that directs incoming calls to the correct department. And manufacturing workers around the country are being replaced in favor of massive industrial machines. These are some of the most common jobs in the country that are already being taken over by automation. Another commonly used example of an occupation that is primed to be automated is truck driving. There are currently 3.5 million truck drivers in America, and it is the most common job in 29 states. With companies like Tesla already releasing autonomous vehicles in the consumer market, it is not long before they develop similar technologies for commercial use. Once that technology is developed it will be impossible for human drivers to compete with automated trucks that do not need to stop for sleep or food \cite{rogan_2019}. With technology only improving over the next decade, the future is grim for millions of Americans who will lose their jobs to automation. While some argue that this is not the first time in history technological advancements have made jobs obsolete, and that new jobs are always created to replace the old ones. There is something distinctly different this time, and that is the immense power and reach technology has today. There has never been a catalyst for change like the technology of the 21st century, and its impact across so many industries could potentially have devastating consequences on the country's workforce \cite{the_social_dilemma_2020}.

\section{The Population at Greatest Risk of being Adversely Impacted by Automation}
A trademark of every industrial revolution is the new jobs that are created as a result of new technologies being implemented to improve existing operations. However, unlike the previous industrial revolutions, the new jobs that are being created by automation are not realistically accessible to the population of displaced workers. Today, the average American is a high school graduate, and of those who do go on to pursue a college degree only 32 percent actually graduate. As a result over half of the American workforce does not have a college degree and works in one of five industries which are in danger of being taken over by automation \cite{rogan_2019}. One way that automation will potentially change different industries is a wide range implementation of Internet of Things technology. As we develop a global infrastructure using existing communication technologies like WiFi, Bluetooth, and 5G networks, companies will begin implementing industrial IoT networks that provide communications between physical machines and virtual programs that slowly eliminate the need for human intervention. Another expected improvement to business operations is large scale implementation of artificial intelligence to replace human workers in many different industries \cite{edseee.911179220200401}. An important note is that automation requires highly skilled workers to organize and monitor these processes. So most physical labor jobs will be replaced with more technologically advanced roles that will require training or certifications to qualify for. Lower skilled jobs will be the first to be adversely impacted with potential repercussions for higher skilled jobs becoming evident only decades later \cite{freeman2001time}. In the past college was cheaper and there were more skilled workers available who were able to learn new skills like how to operate a mill, or a train. In contrast, if the average American today lost their job to automation, it would be much harder for them to transition into the newly created higher skilled jobs \cite{rider_2021}. Not only does this population lack the resources and opportunities required to learn new skills, the aid that is provided to these displaced workers by the government has been historically ineffective. Independent studies done on government funded retraining programs had a success rate of 0 to 15 percent. As a result, roughly 85 percent of displaced workers from manufacturing jobs in the Midwest were unable to learn the new skills required to transition into the careers of the future \cite{rogan_2019}. 

\section{Conclusion}
Now what would happen if half of the American workforce were slowly pushed out of their jobs, forced to leave careers that they have spent years honing their skills for? What would happen if we are only able to successfully train and integrate at best, 15 percent of the displaced workers? What happens to the millions of Americans who would still be out of work and be forced to live on welfare programs? Not considering the myriad of mental issues that can develop due to the fear  of unemployment \cite{3325322720201130} and extended periods of unemployment \cite{PATEL201854}, the way the economy is structured now, without a college degree and ineffective government funded retraining programs, this population will struggle to access the new opportunities that will be created by the rise of automation.
Within the next couple decades every industry will be impacted by automation in one way or another. Even if a job is not completely overtaken by automation, it will become technologically assisted in some way and will require more training or certifications in order for candidates to qualify. Drastic change to education practices and an increase in social awareness is required to avoid a large scale economic crisis \cite{edseee.911179220200401}.


\bibliographystyle{plain}
\bibliography{sources}
\end{document}

